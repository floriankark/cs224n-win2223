\graphicspath{ {images/} }

\titledquestion{Analyzing NMT Systems}[25]

\begin{parts}

    \part[3] Look at the {\monofam{src.vocab}} file for some examples of phrases and words in the source language vocabulary. When encoding an input Mandarin Chinese sequence into ``pieces'' in the vocabulary, the tokenizer maps the sequence to a series of vocabulary items, each consisting of one or more characters (thanks to the {\monofam{sentencepiece}} tokenizer, we can perform this segmentation even when the original text has no white space). Given this information, how could adding a 1D Convolutional layer after the embedding layer and before passing the embeddings into the bidirectional encoder help our NMT system? \textbf{Hint:} each Mandarin Chinese character is either an entire word or a morpheme in a word. Look up the meanings of 电, 脑, and 电脑 separately for an example. The characters 电 (electricity) and  脑 (brain) when combined into the phrase 电脑 mean computer.

    \ifans{Adding a 1D Convolutional layer after the embedding layer and before passing the embeddings into the bidirectional encoder could help the NMT system by allowing the model to capture local dependencies and patterns among the character sequences. Since each Mandarin Chinese character is either an entire word or a morpheme in a word, the convolutional layer can identify these patterns and use them to inform the representation of the word or phrase as a whole. For example, in the case of the characters 电, 脑, and 电脑, the convolutional layer could potentially learn to recognize the pattern of the characters 电 and 脑 occurring together to form the word "computer". By capturing these patterns, the model may be able to improve its ability to handle rare or unseen words, which is important for NMT systems since they must be able to translate sentences containing previously unseen vocabulary.}


    \part[8] Here we present a series of errors we found in the outputs of our NMT model (which is the same as the one you just trained). For each example of a reference (i.e., `gold') English translation, and NMT (i.e., `model') English translation, please:
    
    \begin{enumerate}
        \item Identify the error in the NMT translation.
        \item Provide possible reason(s) why the model may have made the error (either due to a specific linguistic construct or a specific model limitation).
        \item Describe one possible way we might alter the NMT system to fix the observed error. There are more than one possible fixes for an error. For example, it could be tweaking the size of the hidden layers or changing the attention mechanism.
    \end{enumerate}
    
    Below are the translations that you should analyze as described above. Only analyze the underlined error in each sentence. Rest assured that you don't need to know Mandarin to answer these questions. You just need to know English! If, however, you would like some additional color on the source sentences, feel free to use a resource like \url{https://www.archchinese.com/chinese_english_dictionary.html} to look up words. Feel free to search the training data file to have a better sense of how often certain characters occur.

    \begin{subparts}
        \subpart[2]
        \textbf{Source Sentence:} 贼人其后被警方拘捕及被判处盗窃罪名成立。 \newline
        \textbf{Reference Translation:} \textit{\underline{the culprits were} subsequently arrested and convicted.}\newline
        \textbf{NMT Translation:} \textit{\underline{the culprit was} subsequently arrested and sentenced to theft.}
        
        \ifans{
        \begin{enumerate}
        \item The error in the NMT translation is the use of the singular form "culprit" instead of the plural "culprits".
        \item The model may have made this error because of a lack of attention to the plural form of the noun "culprits". Additionally, the model may have been trained on data that had a higher frequency of singular nouns than plural nouns.
        \item One possible way to fix this error is to increase the weight of the attention mechanism on the number of nouns in the source sentence or to increase the frequency of plural nouns in the training data.
        \end{enumerate}
        }


        \subpart[2]
        \textbf{Source Sentence}: 几乎已经没有地方容纳这些人,资源已经用尽。\newline
        \textbf{Reference Translation}: \textit{there is almost no space to accommodate these people, and resources have run out.   }\newline
        \textbf{NMT Translation}: \textit{the resources have been exhausted and \underline{resources have been exhausted}.}
        
        \ifans{
        \begin{enumerate}
        \item The error in the NMT translation is that the word "space" is missing from the translation, and the word "resources" is repeated.
        \item The model may have made the error because it did not accurately capture the meaning of the word "容纳" (rǒngnà), which means "to accommodate" or "to contain". Additionally, the repetition of the word "resources" may be due to an error in the model's attention mechanism.
        \item One possible way to fix this error is to adjust the attention mechanism to better capture the meaning of the sentence. Additionally, increasing the amount of training data or adjusting the model architecture could also help improve the accuracy of the translation.
        \end{enumerate}
        }

        \subpart[2]
        \textbf{Source Sentence}: 当局已经宣布今天是国殇日。 \newline
        \textbf{Reference Translation}: \textit{authorities have announced \underline{a national mourning today.}}\newline
        \textbf{NMT Translation}: \textit{the administration has announced \underline{today's day.}}
        
        \ifans{
        \begin{enumerate}
        \item The error is that the NMT translation misses the meaning of “国殇日” which means “national mourning day” and mistranslates it as “today’s day”.
        \item The model may not have learned the specific translation of “国殇日” as it is a culturally specific term, and may have instead relied on the literal translation of each individual character.
        \item The model may benefit from being trained on a larger corpus of text that includes more culturally specific terms and phrases. Additionally, the model could be improved by incorporating additional context and domain-specific knowledge during the translation process, such as incorporating knowledge of national holidays and events.
        \end{enumerate}
        }
        
        \subpart[2] 
        \textbf{Source Sentence\footnote{This is a Cantonese sentence! The data used in this assignment comes from GALE Phase 3, which is a compilation of news written in simplified Chinese from various sources scraped from the internet along with their translations. For more details, see \url{https://catalog.ldc.upenn.edu/LDC2017T02}. }:} 俗语有云:``唔做唔错"。\newline
        \textbf{Reference Translation:} \textit{\underline{`` act not, err not "}, so a saying goes.}\newline
        \textbf{NMT Translation:} \textit{as the saying goes, \underline{`` it's not wrong. "}}
        
        \ifans{
        \begin{enumerate}
        \item The error is that the NMT translation is missing the first half of the reference translation, which is the translation of the Chinese idiom.
        \item The model may have difficulty understanding idiomatic expressions, as well as the structure of the Chinese language, which often puts the subject at the beginning of the sentence.
        \item To help the model better understand idiomatic expressions, we could provide it with a larger training set that includes more diverse examples of idioms and their translations. Additionally, we could explore incorporating a pre-trained language model or training the model on a larger corpus of data to improve its understanding of the structure of the Chinese language. 
        \end{enumerate}
        }
    \end{subparts}


    \part[14] BLEU score is the most commonly used automatic evaluation metric for NMT systems. It is usually calculated across the entire test set, but here we will consider BLEU defined for a single example.\footnote{This definition of sentence-level BLEU score matches the \texttt{sentence\_bleu()} function in the \texttt{nltk} Python package. Note that the NLTK function is sensitive to capitalization. In this question, all text is lowercased, so capitalization is irrelevant. \\ \url{http://www.nltk.org/api/nltk.translate.html\#nltk.translate.bleu_score.sentence_bleu}
    } 
    Suppose we have a source sentence $\bs$, a set of $k$ reference translations $\br_1,\dots,\br_k$, and a candidate translation $\bc$. To compute the BLEU score of $\bc$, we first compute the \textit{modified $n$-gram precision} $p_n$ of $\bc$, for each of $n=1,2,3,4$, where $n$ is the $n$ in \href{https://en.wikipedia.org/wiki/N-gram}{n-gram}:
    \begin{align}
        p_n = \frac{ \displaystyle \sum_{\text{ngram} \in \bc} \min \bigg( \max_{i=1,\dots,k} \text{Count}_{\br_i}(\text{ngram}), \enspace \text{Count}_{\bc}(\text{ngram}) \bigg) }{\displaystyle \sum_{\text{ngram}\in \bc} \text{Count}_{\bc}(\text{ngram})}
    \end{align}
     Here, for each of the $n$-grams that appear in the candidate translation $\bc$, we count the maximum number of times it appears in any one reference translation, capped by the number of times it appears in $\bc$ (this is the numerator). We divide this by the number of $n$-grams in $\bc$ (denominator). \newline 

    Next, we compute the \textit{brevity penalty} BP. Let $len(c)$ be the length of $\bc$ and let $len(r)$ be the length of the reference translation that is closest to $len(c)$ (in the case of two equally-close reference translation lengths, choose $len(r)$ as the shorter one). 
    \begin{align}
        BP = 
        \begin{cases}
            1 & \text{if } len(c) \ge len(r) \\
            \exp \big( 1 - \frac{len(r)}{len(c)} \big) & \text{otherwise}
        \end{cases}
    \end{align}
    Lastly, the BLEU score for candidate $\bc$ with respect to $\br_1,\dots,\br_k$ is:
    \begin{align}
        BLEU = BP \times \exp \Big( \sum_{n=1}^4 \lambda_n \log p_n \Big)
    \end{align}
    where $\lambda_1,\lambda_2,\lambda_3,\lambda_4$ are weights that sum to 1. The $\log$ here is natural log.
    \newline
    \begin{subparts}
        \subpart[5] Please consider this example: \newline
        Source Sentence $\bs$: \textbf{需要有充足和可预测的资源。} 
        \newline
        Reference Translation $\br_1$: \textit{resources have to be sufficient and they have to be predictable}
        \newline
        Reference Translation $\br_2$: \textit{adequate and predictable resources are required}
        
        NMT Translation $\bc_1$: there is a need for adequate and predictable resources
        
        NMT Translation $\bc_2$: resources be sufficient and predictable to
        
        Please compute the BLEU scores for $\bc_1$ and $\bc_2$. Let $\lambda_i=0.5$ for $i\in\{1,2\}$ and $\lambda_i=0$ for $i\in\{3,4\}$ (\textbf{this means we ignore 3-grams and 4-grams}, i.e., don't compute $p_3$ or $p_4$). When computing BLEU scores, show your work (i.e., show your computed values for $p_1$, $p_2$, $len(c)$, $len(r)$ and $BP$). Note that the BLEU scores can be expressed between 0 and 1 or between 0 and 100. The code is using the 0 to 100 scale while in this question we are using the \textbf{0 to 1} scale. Please round your responses to 3 decimal places. 
        \newline
        
        Which of the two NMT translations is considered the better translation according to the BLEU Score? Do you agree that it is the better translation?
        
        \ifans{ \\
        BLEU score for c1: \\
        $p_1 = 4/9$ \\
        $p_2 = 3/8$ \\
        $p_3 = 2/7$ \\
        $p_4 = 1/6$ \\
        $len(c_1) = 9$ \\
        $len(r_2) = 6$ \\
        $BP = 1$ \\
        $BLEU = BP * exp(0.5log(p_1) + 0.5log(p_2)) = 1 * exp(0.5log(4/9) + 0.5log(3/8)) = 0.408$ \\
        \\
        BLEU score for c2:\\
        $p_1 = 6/6 = 1$ \\
        $p_2 = 2/5$ \\
        $p_3 = 1/4$ \\
        $len(c_2) = 6$ \\
        $len(r_2) = 6$ \\
        $BP = 1$ \\
        $BLEU = BP * exp(0.5log(p_1) + 0.5log(p_2)) = 1 * exp(0.5log(1) + 0.5log(2/5)) = 0.632$  \\
        \\
        According to the BLEU score, c2 is the better translation with a score of 0.632 compared to 0.408 for c2. However I would not agree that c2 is a better translation.
        }
        
        \subpart[5] Our hard drive was corrupted and we lost Reference Translation $\br_1$. Please recompute BLEU scores for $\bc_1$ and $\bc_2$, this time with respect to $\br_2$ only. Which of the two NMT translations now receives the higher BLEU score? Do you agree that it is the better translation?
        
        \ifans{ \\
        BLEU score for c1: \\
        $p_1 = 4/9$ \\
        $p_2 = 3/8$ \\
        $len(c_1) = 9$ \\
        $len(r_2) = 6$ \\
        $BP = 1$ \\
        $BLEU = BP * exp(0.5log(p_1) + 0.5log(p_2)) = 1 * exp(0.5log(4/9) + 0.5log(3/8)) = 0.408$ \\
        \\
        BLEU score for c2:\\
        $p_1 = 3/6 = 1$ \\
        $p_2 = 1/5$ \\
        $len(c_2) = 6$ \\
        $len(r_2) = 6$ \\
        $BP = 1$ \\
        $BLEU = BP * exp(0.5log(p_1) + 0.5log(p_2)) = 1 * exp(0.5log(3/6) + 0.5log(1/5)) = 0.316$  \\
        \\
        This time, c1 is the better translation with an unchanged score of 0.408 compared to the new score of 0.316 for c2. This time I would agree that c1 is the better translation.
        }
        
        \subpart[2] Due to data availability, NMT systems are often evaluated with respect to only a single reference translation. Please explain (in a few sentences) why this may be problematic. In your explanation, discuss how the BLEU score metric assesses the quality of NMT translations when there are multiple reference transitions versus a single reference translation.
        
        \ifans{
        Evaluating NMT systems with respect to only a single reference translation can be problematic because translations can be subjective, and different translators may produce different translations for the same source sentence. Therefore, a single reference translation may not necessarily represent the true meaning and intention of the source sentence, and it may not capture the variability and diversity of the target language.

        When there are multiple reference translations, it can provide a more robust and reliable evaluation, as it takes into account the variability and diversity of the target language and reduces the bias and uncertainty of a single reference translation.
        }
        
        \subpart[2] List two advantages and two disadvantages of BLEU, compared to human evaluation, as an evaluation metric for Machine Translation. 
        
        \ifans{ \\
        \underline{\textbf{Advantages}} \\
        \textbf{Objectivity}: BLEU provides an objective and quantitative measure of translation quality, which can be computed automatically without human involvement. This can save time and resources, and it can reduce the subjectivity and bias of human evaluation.\\
        \textbf{Scalability}: BLEU can be applied to large datasets with multiple translations, which can enable a more comprehensive and representative evaluation. \\
        \underline{\textbf{Disadvantages}} \\
        \textbf{Quality and Adequacy}: BLEU measures only lexical and n-gram similarities between the reference and translated sentences, and it does not capture other aspects of translation quality such as fluency, coherence, and idiomatic expressions. Therefore, BLEU may not reflect the true quality and adequacy of the translation. \\
        \textbf{Meaning and Context}: BLEU does not consider the meaning and context of the source and target sentences, and it may assign high scores to translations that are not faithful to the original meaning or that produce nonsense or irrelevant sentences.
        }
        
    \end{subparts}
\end{parts}
